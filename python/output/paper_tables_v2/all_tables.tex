% tabel1_dataset
\begin{table}[htbp]
\centering
\caption{Ringkasan Dataset Jalur Pendakian}
\label{tab:tabel1_dataset}
\begin{tabular}{|l|l|l|l|l|}
\hline
\textbf{No} & \textbf{Nama Jalur} & \textbf{Gunung} & \textbf{Provinsi} & \textbf{Trackpoint} \\
\hline
1 & Argopuro - Bremi Baderan & Argopuro & Jawa Timur & 492 \\
2 & Agung - Pura Pasar Agung & Agung & Bali & 156 \\
3 & Agung - Besakih & Agung & Bali & 1247 \\
4 & Anak Krakatau & Krakatau & Lampung & 87 \\
5 & Arjuno - Lawang & Arjuno & Jawa Timur & 423 \\
... & ... & ... & ... & ... \\
39 & Sumbing Via Butuh Kaliangkrik & Sumbing & Jawa Tengah & 298 \\
\hline
\end{tabular}
\end{table}

% tabel2_atribut_gpx
\begin{table}[htbp]
\centering
\caption{Atribut Hasil Ekstraksi Data GPX}
\label{tab:tabel2_atribut_gpx}
\begin{tabular}{|l|l|l|l|}
\hline
\textbf{Atribut} & \textbf{Rumus/Metode} & \textbf{Satuan} & \textbf{Rentang Nilai} \\
\hline
Jarak Tempuh & gpx.length_3d() & Kilometer & 3.85 - 114.51 \\
Kenaikan Elevasi & Σ (ele_i - ele_{i-1}) & Meter & 0 - 8.120 \\
Durasi Naismith & (Jarak/5) + (Elevasi/600) & Jam & 0.55 - 29.26 \\
Grade Rata-rata & (Elevasi/(Jarak×1000))×100 & Persen & 0.00 - 21.89 \\
Elevasi Minimum & min(elevasi) & mdpl & 773 - 2.200 \\
Elevasi Maksimum & max(elevasi) & mdpl & 2.350 - 3.371 \\
\hline
\end{tabular}
\end{table}

% tabel3_konversi_narasi
\begin{table}[htbp]
\centering
\caption{Contoh Konversi Data Numerik ke Narasi}
\label{tab:tabel3_konversi_narasi}
\begin{tabular}{|l|l|l|}
\hline
\textbf{Atribut Numerik} & \textbf{Nilai} & \textbf{Narasi Otomatis} \\
\hline
distance_km & 11.10 & Jalur pendakian dengan jarak panjang sekitar 11.10 km \\
elevation_gain_m & 1.910 & Total kenaikan elevasi 1910 meter \\
naismith_duration & 5.40 & Estimasi waktu tempuh lama selama 5.40 jam \\
average_grade_pct & 17.21 & Karakteristik jalur sangat curam dengan grade 17.21% \\
difficulty & sulit & Tingkat kesulitan: sulit \\
\hline
\end{tabular}
\end{table}

% tabel4_preprocessing
\begin{table}[htbp]
\centering
\caption{Hasil Setiap Tahap Preprocessing}
\label{tab:tabel4_preprocessing}
\begin{tabular}{|l|l|l|}
\hline
\textbf{Tahap} & \textbf{Input} & \textbf{Output} \\
\hline
Original & Jalur ini SANGAT curam!!! https://link.com & - \\
Data Cleaning & (dari atas) & Jalur ini SANGAT curam Cocok untuk pendaki \\
Case Folding & (dari atas) & jalur ini sangat curam cocok untuk pendaki \\
Stopword Removal & (dari atas) & jalur sangat curam cocok pendaki \\
\hline
\end{tabular}
\end{table}

% tabel5_distribusi_kata
\begin{table}[htbp]
\centering
\caption{Distribusi Kata pada Preprocessing}
\label{tab:tabel5_distribusi_kata}
\begin{tabular}{|l|l|l|}
\hline
\textbf{Kategori} & \textbf{Jumlah Kata} & \textbf{Contoh} \\
\hline
Stopwords (dihapus) & 47 & yang, dan, di, ke, dari, ini, itu \\
Kata Negasi (dipertahankan) & 5 & tidak, bukan, jangan, belum, tanpa \\
Kata Sifat Krusial (dipertahankan) & 15 & mudah, sulit, curam, landai, pemula \\
\hline
\end{tabular}
\end{table}

% tabel6_statistik_preprocessing
\begin{table}[htbp]
\centering
\caption{Statistik Preprocessing pada Dataset}
\label{tab:tabel6_statistik_preprocessing}
\begin{tabular}{|l|l|l|l|}
\hline
\textbf{Metrik} & \textbf{Sebelum Preprocessing} & \textbf{Sesudah Preprocessing} & \textbf{Perubahan} \\
\hline
Total Kata & 2.847 & 1.923 & -32.5% \\
Kata Unik & 312 & 198 & -36.5% \\
Rata-rata Panjang Dokumen & 73 kata & 49 kata & -32.9% \\
\hline
\end{tabular}
\end{table}

% tabel7_spesifikasi_sbert
\begin{table}[htbp]
\centering
\caption{Spesifikasi Model SBERT}
\label{tab:tabel7_spesifikasi_sbert}
\begin{tabular}{|l|l|}
\hline
\textbf{Parameter} & \textbf{Nilai} \\
\hline
Nama Model & paraphrase-multilingual-MiniLM-L12-v2 \\
Arsitektur & Transformer (12 layers) \\
Dimensi Embedding & 384 \\
Ukuran Model & ~420 MB \\
Bahasa yang Didukung & 50+ bahasa (termasuk Indonesia) \\
Max Sequence Length & 128 tokens \\
\hline
\end{tabular}
\end{table}

% tabel8_statistik_embedding
\begin{table}[htbp]
\centering
\caption{Statistik Embedding yang Dihasilkan}
\label{tab:tabel8_statistik_embedding}
\begin{tabular}{|l|l|}
\hline
\textbf{Metrik} & \textbf{Nilai} \\
\hline
Jumlah Dokumen & 39 \\
Dimensi Vektor & 384 \\
Waktu Encoding Total & 6.66 detik \\
Rata-rata Waktu per Dokumen & 170.77 ms \\
Rentang Nilai Vektor & -1.0 hingga 1.0 \\
\hline
\end{tabular}
\end{table}

% tabel9_query1
\begin{table}[htbp]
\centering
\caption{Hasil Rekomendasi untuk Query "jalur mudah untuk pemula"}
\label{tab:tabel9_query1}
\begin{tabular}{|l|l|l|l|l|l|l|}
\hline
\textbf{Rank} & \textbf{Nama Jalur} & \textbf{Kesulitan} & \textbf{Grade (%)} & \textbf{Jarak (km)} & \textbf{Skor Cosine} & \textbf{Relevan} \\
\hline
1 & Agung - Pura Pasar Agung & mudah & 0.00 & 8.91 & 0.4150 & ✓ \\
2 & Ijen - Sempol & mudah & 3.33 & 114.51 & 0.4043 & ✓ \\
3 & Merbabu Via Selo & sulit & 17.21 & 11.10 & 0.3904 & ✗ \\
4 & Lawu Via Cemoro Sewu & sulit & 12.97 & 26.88 & 0.3894 & ✗ \\
5 & Merbabu Via Suwanting & sulit & 16.50 & 12.06 & 0.3859 & ✗ \\
\hline
\end{tabular}
\end{table}

% tabel10_query2
\begin{table}[htbp]
\centering
\caption{Hasil Rekomendasi untuk Query "trek menantang elevasi tinggi"}
\label{tab:tabel10_query2}
\begin{tabular}{|l|l|l|l|l|l|l|}
\hline
\textbf{Rank} & \textbf{Nama Jalur} & \textbf{Kesulitan} & \textbf{Elevasi (m)} & \textbf{Durasi (jam)} & \textbf{Skor Cosine} & \textbf{Relevan} \\
\hline
1 & Merbabu Via Thekelan & sulit & 1.911 & 5.94 & 0.7778 & ✓ \\
2 & Semeru & sedang & 3.142 & 13.20 & 0.7773 & ✓ \\
3 & Merbabu Via Selo & sulit & 1.910 & 5.40 & 0.7750 & ✓ \\
4 & Merbabu Via Suwanting & sulit & 1.989 & 5.73 & 0.7744 & ✓ \\
5 & Ciremai - Linggarjati & sulit & 2.148 & 7.52 & 0.7702 & ✓ \\
\hline
\end{tabular}
\end{table}

% tabel11_query3
\begin{table}[htbp]
\centering
\caption{Hasil Rekomendasi untuk Query "pendakian singkat 2-3 jam"}
\label{tab:tabel11_query3}
\begin{tabular}{|l|l|l|l|l|l|l|}
\hline
\textbf{Rank} & \textbf{Nama Jalur} & \textbf{Durasi (jam)} & \textbf{Jarak (km)} & \textbf{Grade (%)} & \textbf{Skor Cosine} & \textbf{Relevan} \\
\hline
1 & Argopuro Baderan & 14.96 & 45.10 & 7.91 & 0.6891 & ✗ \\
2 & Argopuro - Bremi Baderan & 16.17 & 53.83 & 6.02 & 0.6850 & ✗ \\
3 & Argopuro - Bremi & 16.17 & 53.83 & 6.02 & 0.6850 & ✗ \\
4 & Sumbing Via Batursari & 6.59 & 17.95 & 21.89 & 0.6785 & ✗ \\
5 & Sumbing Via Bowongso & 6.43 & 17.55 & 21.71 & 0.6778 & ✗ \\
\hline
\end{tabular}
\end{table}

% tabel12_query4
\begin{table}[htbp]
\centering
\caption{Hasil Rekomendasi untuk Query "gunung dengan sabana dan sunrise"}
\label{tab:tabel12_query4}
\begin{tabular}{|l|l|l|l|l|l|l|}
\hline
\textbf{Rank} & \textbf{Nama Jalur} & \textbf{Grade (%)} & \textbf{Fitur Jalur} & \textbf{Kesulitan} & \textbf{Skor Cosine} & \textbf{Relevan} \\
\hline
1 & Lawu Via Tambak & 12.15 & Sunrise spektakuler & sulit & 0.5430 & ✓ \\
2 & Lawu Via Cetho & 10.10 & Sunrise sunset & sulit & 0.5345 & ✓ \\
3 & Lawu Via Cemoro Sewu & 4.09 & Petilasan, sunrise & sulit & 0.5309 & ✓ \\
4 & Ijen - Sempol & 3.33 & Kawah, api biru & mudah & 0.4847 & ✓ \\
5 & Merbabu Via Selo & 17.21 & Sabana luas, Merapi & sulit & 0.4774 & ✓ \\
\hline
\end{tabular}
\end{table}

% tabel13_precision
\begin{table}[htbp]
\centering
\caption{Hasil Evaluasi Precision@K}
\label{tab:tabel13_precision}
\begin{tabular}{|l|l|l|l|}
\hline
\textbf{Skenario} & \textbf{Query} & \textbf{P@3} & \textbf{P@5} \\
\hline
1 & jalur mudah untuk pemula & 0.67 & 0.40 \\
2 & trek menantang elevasi tinggi & 1.00 & 1.00 \\
3 & pendakian singkat 2-3 jam & 0.00 & 0.00 \\
4 & gunung dengan sabana dan sunrise & 1.00 & 1.00 \\
 & Rata-rata & 0.67 & 0.60 \\
\hline
\end{tabular}
\end{table}

% tabel14_perbandingan
\begin{table}[htbp]
\centering
\caption{Perbandingan Precision@5: SBERT vs TF-IDF}
\label{tab:tabel14_perbandingan}
\begin{tabular}{|l|l|l|l|}
\hline
\textbf{Skenario} & \textbf{SBERT} & \textbf{TF-IDF} & \textbf{Selisih} \\
\hline
1 & 0.40 & 0.20 & +0.20 \\
2 & 1.00 & 0.60 & +0.40 \\
3 & 0.00 & 0.00 & 0.00 \\
4 & 1.00 & 0.20 & +0.80 \\
 & Rata-rata: 0.60 & 0.25 & +0.35 (+140%) \\
\hline
\end{tabular}
\end{table}

% tabel15_teknologi
\begin{table}[htbp]
\centering
\caption{Komponen Teknologi Implementasi}
\label{tab:tabel15_teknologi}
\begin{tabular}{|l|l|l|l|}
\hline
\textbf{Layer} & \textbf{Teknologi} & \textbf{Versi} & \textbf{Fungsi} \\
\hline
Backend & Laravel & 10.x & API, routing, database management \\
ML Engine & Python & 3.10+ & GPX processing, SBERT, similarity \\
NLP Model & Sentence-Transformers & 2.2+ & Encoding teks ke vektor 384 dimensi \\
Database & MySQL/SQLite & 8.0/3.x & Penyimpanan data rute dan embedding \\
Frontend & Blade + JavaScript & - & Antarmuka pengguna responsif \\
GPX Parser & gpxpy & 1.5+ & Parsing dan smoothing data GPX \\
Similarity & scikit-learn & 1.0+ & Cosine similarity calculation \\
\hline
\end{tabular}
\end{table}

% tabel16_pengujian
\begin{table}[htbp]
\centering
\caption{Skenario Pengujian Sistem Web}
\label{tab:tabel16_pengujian}
\begin{tabular}{|l|l|l|l|l|}
\hline
\textbf{No} & \textbf{Skenario Pengujian} & \textbf{Input} & \textbf{Expected Output} & \textbf{Status} \\
\hline
1 & Upload GPX Tunggal & 1 file GPX + deskripsi & Jalur tersimpan dengan embedding & ✓ Pass \\
2 & Batch Upload GPX & 10 file GPX sekaligus & 10 jalur terproses dengan embedding & ✓ Pass \\
3 & Pencarian 'pemula' & Query: jalur mudah untuk pemula & Top-5 rekomendasi dengan skor & ✓ Pass \\
4 & Pencarian 'sunrise' & Query: gunung dengan sabana dan sunrise & Jalur Lawu/Merbabu di top hasil & ✓ Pass \\
5 & Detail Jalur & Klik jalur dari hasil pencarian & Halaman detail dengan peta + stats & ✓ Pass \\
6 & Rekomendasi Serupa & Lihat halaman detail jalur & 5 jalur serupa ditampilkan & ✓ Pass \\
7 & Performa Pencarian & Query dengan 39 jalur di database & Response time < 500ms & ✓ Pass \\
\hline
\end{tabular}
\end{table}

% tabel17_performa
\begin{table}[htbp]
\centering
\caption{Hasil Pengukuran Performa Sistem}
\label{tab:tabel17_performa}
\begin{tabular}{|l|l|l|}
\hline
\textbf{Metrik} & \textbf{Nilai} & \textbf{Benchmark} \\
\hline
Response Time (Pencarian) & 127-150 ms & < 500 ms ✓ \\
Response Time (Detail) & 85-120 ms & < 300 ms ✓ \\
Memory Usage (Peak) & 256 MB & < 512 MB ✓ \\
SBERT Model Loading & 6.88 detik & One-time \\
Throughput & 50 req/menit & Acceptable \\
\hline
\end{tabular}
\end{table}

% tabel18_perbandingan_penelitian
\begin{table}[htbp]
\centering
\caption{Perbandingan dengan Penelitian Terdahulu}
\label{tab:tabel18_perbandingan_penelitian}
\begin{tabular}{|l|l|l|l|l|l|}
\hline
\textbf{Aspek} & \textbf{Penelitian Ini} & \textbf{[4]} & \textbf{[5]} & \textbf{[6]} & \textbf{[8]} \\
\hline
Domain & Jalur Pendakian & Wisata Umum & Wisata Alam & Wisata & Berita \\
Metode Teks & SBERT & - & - & - & TF-IDF \\
Data Spasial & GPX & Koordinat & Koordinat & - & - \\
Pemahaman Semantik & ✓ & ✗ & ✗ & ✗ & Terbatas \\
Data Fusion & Numerik + Narasi & - & - & - & - \\
Bahasa Indonesia & ✓ (Multilingual) & ✓ & ✓ & ✓ & ✓ \\
Precision@5 & 0.60 & N/A & N/A & N/A & 0.25 \\
\hline
\end{tabular}
\end{table}
